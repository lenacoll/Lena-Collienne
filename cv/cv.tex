%%%%%%%%%%%%%%%%%%%%%%%%%%%%%%%%%%%%%%%%%
% "ModernCV" CV and Cover Letter
% LaTeX Template
% Version 1.3 (29/10/16)
%
% This template has been downloaded from:
% http://www.LaTeXTemplates.com
%
% Original author:
% Xavier Danaux (xdanaux@gmail.com) with modifications by:
% Vel (vel@latextemplates.com)
%
% License:
% CC BY-NC-SA 3.0 (http://creativecommons.org/licenses/by-nc-sa/3.0/)
%
% Important note:
% This template requires the moderncv.cls and .sty files to be in the same 
% directory as this .tex file. These files provide the resume style and themes 
% used for structuring the document.
%
%%%%%%%%%%%%%%%%%%%%%%%%%%%%%%%%%%%%%%%%%

%----------------------------------------------------------------------------------------
%	PACKAGES AND OTHER DOCUMENT CONFIGURATIONS
%----------------------------------------------------------------------------------------

\documentclass[11pt,a4paper,sans]{moderncv} % Font sizes: 10, 11, or 12; paper sizes: a4paper, letterpaper, a5paper, legalpaper, executivepaper or landscape; font families: sans or roman

\moderncvstyle{casual} % CV theme - options include: 'casual' (default), 'classic', 'oldstyle' and 'banking'
\moderncvcolor{black} % CV color - options include: 'blue' (default), 'orange', 'green', 'red', 'purple', 'grey' and 'black'

\usepackage{lipsum} % Used for inserting dummy 'Lorem ipsum' text into the template
\usepackage{todonotes}
\usepackage[skins]{tcolorbox}
\usepackage{fontawesome5}

\usepackage[scale=0.75]{geometry} % Reduce document margins
%\setlength{\hintscolumnwidth}{3cm} % Uncomment to change the width of the dates column
%\setlength{\makecvtitlenamewidth}{10cm} % For the 'classic' style, uncomment to adjust the width of the space allocated to your name

%----------------------------------------------------------------------------------------
%	NAME AND CONTACT INFORMATION SECTION
%----------------------------------------------------------------------------------------

\firstname{Lena} % Your first name
\familyname{Collienne} % Your last name

% All information in this block is optional, comment out any lines you don't need
\title{Curriculum Vitae}
% \address{123 Broadway}{City, State 12345}
% \mobile{(000) 111 1111}
% \phone{(000) 111 1112}
% \fax{(000) 111 1113}
% \email{lena@lenacoll.de}
% \homepage{lenacoll.de}{lenacoll.de} % The first argument is the url for the clickable link, the second argument is the url displayed in the template - this allows special characters to be displayed such as the tilde in this example
% \extrainfo{additional information}
% \photo[70pt][0.4pt]{pictures/picture} % The first bracket is the picture height, the second is the thickness of the frame around the picture (0pt for no frame)
% \quote{"A witty and playful quotation" - John Smith}

%----------------------------------------------------------------------------------------

\begin{document}

%----------------------------------------------------------------------------------------
%	COVER LETTER
%----------------------------------------------------------------------------------------

% To remove the cover letter, comment out this entire block

% \clearpage

% \recipient{HR Department}{Corporation\\123 Pleasant Lane\\12345 City, State} % Letter recipient
% \date{\today} % Letter date
% \opening{Dear Sir or Madam,} % Opening greeting
% \closing{Sincerely yours,} % Closing phrase
% \enclosure[Attached]{curriculum vit\ae{}} % List of enclosed documents

% \makelettertitle % Print letter title

% \lipsum[1-2] % Dummy text
% \lipsum[4] % Dummy text

% \makeletterclosing % Print letter signature

% \newpage

%----------------------------------------------------------------------------------------
%	CURRICULUM VITAE
%----------------------------------------------------------------------------------------
\makecvtitle % Print the CV title

\vspace{-2cm}
\tcbset{size=small, sharp corners, colframe=black!50!white, width=4.2cm,box align=top,nobeforeafter}
\begin{tcolorbox}
    \faIcon{envelope} \href{mailto:lena@lenacoll.de}{lena@lenacoll.de}\\
    \faIcon{desktop} \href{https://www.lenacoll.de}{lenacoll.de}\\
    \faIcon{github} \href{https://github.com/lenacoll}{github.com/lenacoll}
\end{tcolorbox}
\vspace{0.5cm}

%----------------------------------------------------------------------------------------
%	MAJOR RESEARCH PROJECT
%----------------------------------------------------------------------------------------

\section{Research}

\cventry{}{Postdoctoral Research}{Fred Hutchinson Cancer Center}{}{\small \newline (Advisor: Frederick Matsen IV, PhD)}{Developing novel methods for phylogenetic inference using machine learning.}

\cventry{}{Postdoctoral Research}{University of Canterbury}{}{\newline\small(Advisor: Alex Gavryushkin, PhD)}{Extending subtree prune and regraft operations to ranked phylogenetic trees and investigating thereby introduced distance measures for phylogenetic time trees.}

\cventry{}{PhD Research}{University of Otago}{}{\newline\small(Advisors: Alex Garvyushkin, PhD, and David Bryant, PhD)}{Introducing and analysing spaces of phylogenetic time trees based on nearest neighbour interchange tree rearrangements.}

\cventry{}{M.Sc. Research}{University of Greifswald}{}{\newline(Advisors: Mareike Fischer, PhD, and Alex Gavryushkin, PhD)}{Establishing properties of ranked nearest neighbour interchange moves between ranked phylogenetic trees}

%----------------------------------------------------------------------------------------
%	EDUCATION SECTION
%----------------------------------------------------------------------------------------

\section{Education}

\cventry{2018--2021}{Doctor of Philosophy}{Computer Science}{University of Otago (NZ)}{}{}  % Arguments not required can be left empty
\cventry{2016--2018}{Master of Science}{Biomathematics}{University of Greifswald (GER)}{}{}
\cventry{2012--2015}{Bachelor of Science}{Biomathematics}{University of Greifswald (GER)}{}{}{}

% \section{PhD Thesis}

% \cvitem{Title}{\emph{Spaces of Phylogenetic Time Trees}}
% \cvitem{Supervisors}{Associate Professor Alex Gavryushkin \& Professor David Bryant}
% \cvitem{Description}{}

%----------------------------------------------------------------------------------------
%	WORK EXPERIENCE SECTION
%----------------------------------------------------------------------------------------

\section{Work Experience}


\cventry{2023--now}{Postdoctoral Research Fellow}{Matsen group}{Fred Hutch Cancer Center (US)}{}{}
\cventry{2022--2023}{Postdoctoral Research Fellow}{BioDS lab, School of Mathematics and Statistics}{University of Canterbury (NZ)}{}{}


\section{Teaching Experience}
\cventry{2023--2024}{Facilitator}{Girls Who Code}{Fred Hutch Cancer Center (Seattle, WA, USA)}{}{}
\cventry{2022}{Lecturer}{STAT211: Random Processes}{University of Canterbury (NZ)}{}{}
\cventry{2019}{Tutor}{COSC341: Theory of Computing}{University of Otago (NZ)}{}{}
\cventry{2015--2016}{Summer Research Project}{University of Auckland (NZ)}{}{}{}




%------------------------------------------------
%----------------------------------------------------------------------------------------
%	AWARDS SECTION
%----------------------------------------------------------------------------------------

\section{Scholarships and Awards}

\cvitem{2022}{\textit{Hatherton Award} (Royal Society of New Zealand)}
\cvitem{2021}{\textit{Exceptional PhD thesis} (Division of Science, University of Otago)}
\cvitem{2018--2021}{\textit{University of Otago Doctoral Scholarship}}
\cvitem{2018}{\textit{Externally Funded Research Grant} (Max Planck Institute Pl\"on)}
\cvitem{2015}{\textit{Summer Research Scholarship} (University of Auckland)}
\cvitem{2015}{\textit{PROMOS Travel Scholarship} (University of Greifswald)}
\cvitem{2014--2015}{\textit{Deutschlandstipendium} (Alfried Krupp von Bohlen und Halbach Foundation/Federal Government of Germany)}


%----------------------------------------------------------------------------------------
%	Additional Activities
%----------------------------------------------------------------------------------------

\section{Additional Activities}

\cvitem{2021}{President of the Otago Computer Science Society (University of Otgao)}
\cvitem{2019--2021}{Member of the Postgraduate Committee (Department of Computer Science, University of Otago)}
\cvitem{2019--2021}{Organising the annual Postgraduate Symposium (Department of Computer Science, University of Otago)}
\cvitem{2019--2021}{Member of Student Council (Institute for Mathematics and Computer Science, University of Greifswald)}

%----------------------------------------------------------------------------------------
%	Talks
%----------------------------------------------------------------------------------------

\section{Talks}

\subsection{Conference Talks}

\cventry{2024}{Evolution 2024}{Montreal (CA)}{}{}{Contributed talk: Instability in Phylogenetic Trees after Taxon
Addition}
\cventry{2023}{SMB meeting 2023}{Columbus (OH, US)}{}{}{Invited minisymposium talk: Spaces of Discrete Time Trees}
\cventry{2022}{Phylomania 2022}{Hobart (AU)}{}{}{Contributed talk: Subtree Prune and Regraft on Ranked Trees}
\cventry{2021}{Phylomania 2021 \textnormal{(Best Student Talk Award)}}{Online}{}{}{Contributed talk: Distances between Phylogenetic Time Trees}
\cventry{2021}{NZ Phylogenomics Meeting}{Akaroa (NZ)}{}{}{Contributed talk: The Space of Discrete Coalescent Trees}
\cventry{2020}{NZ Phylogenomics Meeting}{Waiheke (NZ)}{}{}{Contributed talk: Online Algorithms in Computational Biology}
\cventry{2019}{NZ Phylogenomics Meeting}{Napier (NZ)}{}{}{Contributed talk: The Ranked Nearest Neighbour Interchange Space of Phylogenetic Trees}

\subsection{Invited Seminar Talks}
\cventry{2021}{Departmant of Mathematics}{University of Otago (NZ)}{}{}{The Space of Discrete Coalescent Trees}
\cventry{2020}{Online Seminars on Algorithms and Complexity in Phylogenetics}{Online}{}{}{Computing the Ranked Nearest Neighbour Interchange Distance between Ranked Phylogenetic Trees}
\cventry{2019}{Max Planck Instityte for Mathematics in the Science}{Leipzig (GER)}{}{}{The Ranked Nearest Neighbour Interchange space of phylogenetic trees}
\cventry{2017}{Computational Evolution Group}{ETH Zurich (CH)}{}{}{Discrete Time Trees}

\subsection{Poster Presentations}
\cventry{2024}{ICERM workshop ``Algorithmic Advances and Implementation Challenges: Developing Practical Tools for Phylogenetic Inference''}{Providence (RI, US)}{}{}{Poster: Instability in Phylogenetic Trees after Taxon Addition}
\cventry{2022}{Phylomania 2022}{Hobart (AU)}{}{}{Poster: Subtree Prune and Regraft on Ranked Trees}

\subsection{Other}
\cventry{2022}{School of Mathematics and Statistics}{University of Canterbury (NZ)}{}{}{How to Give a (Good) Talk}
\cventry{2020}{Postgraduate Symposium \textnormal{(1st place Best Presentation Award)}}{University of Otago (NZ)}{}{}{The Complexity of Computing the RNNI Distance between Phylogenetic Trees}
\cventry{2020}{Seminar of Departments of Computer Science and Information Science}{University of Otago (NZ)}{}{}{The Complexity of Computing Nearest Neighbour Interchange Distances between Ranked Phylogenetic Trees}
\cventry{2019}{Postgraduate Symposium \textnormal{(2nd place Best Presentation Award)}}{University of Otago (NZ)}{}{}{Online Algorithms in Computational Biology}

%----------------------------------------------------------------------------------------


%----------------------------------------------------------------------------------------
%	Publications
%----------------------------------------------------------------------------------------

\section{Publications}

\cvitem{}{\textbf{Collienne, L.}, Barker, M., Suchard, MA., \& Matsen IV, FA. (2025). Phylogenetic tree instability after taxon addition: empirical frequency, predictability, and consequences for online inference. \textit{Systematic Biology  syae059. \url{https://doi.org/10.1093/sysbio/syae059}}}

\cvitem{}{\textbf{Collienne, L.}, Whidden, C. \& Gavryushkin, A. (2024). Ranked Subtree Prune and Regraft. \textit{Bulleting of Mathematical Biology  86, 24. \url{https://doi.org/10.1007/s11538-023-01244-2}}}

\cvitem{}{Berling, L., \textbf{Collienne, L.} \& Gavryushkin, A. (2024). Estimating the mean in the space of ranked phylogenetic trees. \textit{Bioinformatics, Volume 40, Issue 8}. \url{https://doi.org/10.1093/bioinformatics/btae514}.}

\cvitem{}{Bouckaert, R., \textbf{Collienne, L.} \& Gavryushkin, A. (2022). Online Bayesian Analysis with BEAST2. \textit{BioRxiv}.}

\cvitem{}{\textbf{Collienne, L.} (2021). Spaces of phylogenetic time trees (Thesis, Doctor of Philosophy). University of Otago. Retrieved from \url{http://hdl.handle.net/10523/12606}}

\cvitem{}{\textbf{Collienne, L.}, Elmes, K., Fischer, M., Bryant, D. \& Gavryushkin, A. (2021). Discrete Coalescent Trees. \textit{Journal of Mathematical Biology 83.5, p. 60. issn: 1432-1416.}}

\cvitem{}{\textbf{Collienne, L.} \& Gavryushkin, A. (2021). Computing nearest neighbour interchange distances between ranked phylogenetic trees. \textit{Journal of Mathematical Biology 82.1, p. 8. issn: 1432-1416}.}

%----------------------------------------------------------------------------------------



\end{document}